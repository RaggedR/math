\documentclass{article}
\usepackage{amsmath}
\usepackage{amssymb}

\begin{document}

\title{Orthogonal Projection Proof}
\author{}
\date{}
\maketitle
\Large

\section*{Definition}
A matrix $P \in \mathbb{R}^{n \times n}$ is an \textbf{orthogonal projection} if for any $\mathbf{x} \in \mathbb{R}^n$:
\begin{enumerate}
    \item $P\mathbf{x}$ is in the subspace $W$
    \item $\mathbf{x} - P\mathbf{x}$ is orthogonal to $W$
    \item $P^2 = P$ (idempotent)
\end{enumerate}

\section*{Proof that $P^T = P$}

Let $P$ be an orthogonal projection matrix. For any vectors $\mathbf{u}, \mathbf{v} \in \mathbb{R}^n$:

Since $P\mathbf{u} \in W$ and $\mathbf{v} - P\mathbf{v} \in W^\perp$, their dot product is zero:
\[
(P\mathbf{u}) \cdot (\mathbf{v} - P\mathbf{v}) = 0
\]

In matrix form (using $\mathbf{a} \cdot \mathbf{b} = \mathbf{a}^T\mathbf{b}$):
\[
(P\mathbf{u})^T(\mathbf{v} - P\mathbf{v}) = 0
\]

Expand the expression:
\[
\mathbf{u}^T P^T \mathbf{v} - \mathbf{u}^T P^T P \mathbf{v} = 0
\]

Factor out $\mathbf{u}^T$ and $\mathbf{v}$:
\[
\mathbf{u}^T (P^T - P^T P) \mathbf{v} = 0
\]

Since this holds for \textbf{all} $\mathbf{u}, \mathbf{v}$, we must have:
\[
P^T - P^T P = 0 \quad \Rightarrow \quad P^T = P^T P \quad (1)
\]

Now consider the reverse case: $\mathbf{u} - P\mathbf{u} \in W^\perp$ and $P\mathbf{v} \in W$, so:
\[
(\mathbf{u} - P\mathbf{u}) \cdot (P\mathbf{v}) = 0
\]

In matrix form:
\[
(\mathbf{u} - P\mathbf{u})^T (P\mathbf{v}) = 0
\]

Expand:
\[
\mathbf{u}^T P \mathbf{v} - (P\mathbf{u})^T P \mathbf{v} = 0
\]

\[
\mathbf{u}^T P \mathbf{v} - \mathbf{u}^T P^T P \mathbf{v} = 0
\]

\[
\mathbf{u}^T (P - P^T P) \mathbf{v} = 0
\]

Since this holds for all $\mathbf{u}, \mathbf{v}$:
\[
P - P^T P = 0 \quad \Rightarrow \quad P = P^T P \quad (2)
\]

From (1) and (2):
\[
P^T = P^T P = P
\]

Therefore:
\[
\boxed{P^T = P}
\]

\textbf{Q.E.D.}

\end{document}
