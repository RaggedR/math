\documentclass[11pt]{article}
\usepackage{amsmath, amssymb, amsthm}
\usepackage{booktabs}
\usepackage{geometry}
\usepackage{enumitem}
\usepackage{caption}
\usepackage{mathtools}
\usepackage{fontawesome5}
\usepackage{hyperref}  % MUST be loaded for \texorpdfstring
\geometry{margin=1in}

\title{Mutual Information of Dependent Random Variables}
\author{}
\date{}

\begin{document}

\maketitle

\Large

\section{Probability Theory}

In this section we give a quick recap of some important ideas from probability theory. We will be focusing on the discrete case.

\subsection{Events and Random Variables}


The first important point is to distinguish between an \emph{event} in some sample space $\Omega$ and a \emph{random variable} $X: \Omega \to \mathbb{R}$ which assigns real values to outcomes.


Let $\Omega$ denote the sample space of all possible outcomes of an experiment. An \textbf{event} is any subset of $\Omega$ . It represents a collection of outcomes about which we are interested. 

For example, consider tossing a fair coin twice. The sample space is
\[
\Omega = \{ HH, HT, TH, TT \}.
\]
The event
$
A = \{\text{exactly one head}\} = \{ HT, TH \}
$
is a subset of $\Omega$. Since the coins are both fair, the probability of this event is:
\[
P(A) = P(\{HT, TH\}) = \frac{1}{4} + \frac{1}{4} = \frac{1}{2}.
\]

A \textbf{random variable} $X$ is a function
\[
X : \Omega \to \mathbb{R}
\]
that assigns a numerical value to each outcome in the sample space. When we write $X = x$, we are referring to the \emph{event} consisting of all outcomes $\omega \in \Omega$ for which $X(\omega) = x$:
\[
\{ \omega \in \Omega : X(\omega) = x \}.
\]
For example, let $X = \#\text{heads in two tosses}$. Then
\[
X(HH) = 2, \quad X(HT) = 1, \quad X(TH) = 1, \quad X(TT) = 0.
\]
The statement $X = 1$ corresponds to the event
\[
\{\omega \in \Omega : X(\omega) = 1\} = \{ HT, TH \},
\]
which is exactly the same event as $A$ in the previous example. Thus, a random variable taking on a particular value is just a special kind of event.

We may use the notation: \[P_X(1) = \frac{1}{2}\]

To summarize: An \textbf{event} is a subset of the sample space, representing outcomes of interest. A \textbf{random variable taking a value} defines an event consisting of all outcomes that map to that value. 


\subsection{Conditional Probability, Independence and Bayes Theorem}


The \textbf{conditional probability} of $A$ given $B$ is defined as:
\[
P(A \mid B) = \frac{P(A \cap B)}{P(B)}, \quad \text{for } P(B) > 0.
\]
Equivalently,
\[
P(B \mid A) = \frac{P(A \cap B)}{P(A)}, \quad \text{for } P(A) > 0.
\]


Setting the two different expressions for $P(A \cap B)$ equal to each other in the above equations gives us \textbf{Bayes' theorem}:
\[
\boxed{
P(A \mid B) = \frac{P(B \mid A) \, P(A)}{P(B)}, \quad \text{for } P(B) > 0
}
\]


For random variables $X$ and $Y$ the notation for conditional probability looks like:
    \[
    P_{X \mid Y}(x \mid y) = \frac{P_{X,Y}(x,y)}{P_Y(y)}, \quad P_Y(y) > 0.
    \]
    In this case, $X=x$ and $Y=y$ are themselves events in $\Omega$, and $P_{X \mid Y}(x \mid y)$ measures the probability that $X$ takes value $x$ given that $Y$ takes value $y$.
    
    In this notation Bayes' theorem looks like:
    \begin{equation*} \boxed{
P_{X \mid Y}(x \mid y) = \frac{P_{Y \mid X}(y \mid x) \, P_X(x)}{P_Y(y)} }
\end{equation*}


Two \textbf{events} $A$ and $B$ are \emph{independent} if:
\[
P(A \cap B) = P(A) \, P(B)
\]
Otherwise, for dependent events, we have:
\[
P(A \cap B) = P(A) \, P(B \mid A) = P(B) \, P(A \mid B)
\]
Equivalently, $A$ and $B$ are independent if knowing one tells you nothing about the other:
\[
P(B) = P(B \mid A) \quad \text{or equivalently} \quad P(A) = P(A \mid B)
\]






\subsection{Example of Dependent Events}

Consider rolling two fair six-sided dice.

Define the events:
\[
A = \{\text{sum of the dice is even}\}, \quad
B = \{\text{sum of the dice is greater than 7}\}
\]
There are 36 equally likely outcomes.
The probabilities are calculated as follows:
\begin{itemize}
    \item $P(A) = \frac{18}{36} = 0.5$
    \item $P(B) = \frac{15}{36} \approx 0.4167$
    \item $P(A \cap B) = \frac{9}{36} = 0.25$.
\end{itemize}
To check for independence, we compute the product:
\[
P(A)P(B) = 0.5 \cdot \frac{15}{36} \approx 0.2083
\]
Since $0.25 \neq 0.2083$, $P(A \cap B) \neq P(A)P(B)$, and the events $A$ and $B$ are \textbf{dependent}.

The conditional probability $P(A \mid B) = \frac{P(A \cap B)}{P(B)} = \frac{0.25}{0.4167} \approx 0.6$. 

Since $P(A \mid B) > P(A)$, knowing that the sum is greater than 7 increases the probability that the sum is even, confirming dependence.



\subsection{Another Example of Dependent Events}
Define the events:
\[
C = \{\text{at least one die shows 6}\}, \quad
D = \{\text{sum of the dice is greater than 10}\}
\]
The probabilities are:
\begin{itemize}
    \item $P(C) = \frac{11}{36} \approx 0.3056$
    \item $P(D) = \frac{3}{36} \approx 0.0833$
    \item $P(C \cap D) = \frac{3}{36} \approx 0.0833$.
\end{itemize}
To check for independence, we compare $P(C \cap D)$ with $P(C)P(D)$:
\[
P(C)P(D) = \frac{11}{36} \cdot \frac{3}{36} = \frac{33}{1296} \approx 0.0254
\]
Since $0.0833 \neq 0.0254$, $P(C \cap D) \neq P(C)P(D)$, and the events $C$ and $D$ are \textbf{dependent}.

The conditional probability $P(D \mid C) = \frac{P(C \cap D)}{P(C)} = \frac{0.0833}{0.3056} \approx 0.2727$. 

Since $P(D \mid C) > P(D)$, knowing that one die is a 6 significantly increases the probability of a sum greater than 10, demonstrating dependence.













\section{Coin Tossing Experiment}

We toss a fair coin \textbf{twice}. The four possible \textbf{elementary outcomes} (or \emph{events}) are:
\begin{itemize}
    \item \textbf{HH}: heads then heads  
    \item \textbf{HT}: heads then tails  
    \item \textbf{TH}: tails then heads  
    \item \textbf{TT}: tails then tails
    \end{itemize}
Each occurs with probability $ \frac{1}{4} $, since the coin is fair and tosses are independent.
From these raw outcomes, we define two \textbf{random variables}—that is, numerical summaries:

\begin{itemize}
    \item Let \textbf{$X$} be the difference between the number of heads and tails:
    \[
    X = (\#\text{heads}) - (\#\text{tails})
    \]
    So:
        \item HH $\to X = 2 - 0 = +2$
        \item HT or TH $\to X = 1 - 1 = 0$
        \item TT $\to X = 0 - 2 = -2$

    \item Let \textbf{$Y$} indicate whether the two tosses \textbf{match}:
    \[
    Y = 
    \begin{cases}
    +1 & \text{if both tosses are the same (HH or TT)} \\
    -1 & \text{if they differ (HT or TH)}
    \end{cases}
    \]
\end{itemize}

It is important not to confuse the \emph{outcomes} and the \emph{random variables}. The \textbf{underlying events} are HH, HT, TH, TT, the \textbf{random variables} $X$ and $Y$ assign numbers to the underlying events in such a way that multiple events can lead to the same $(X,Y)$ pair.

\begin{center}
\begin{tabular}{lccc}
\toprule
Event & $X$ & $Y$ & Probability \\
\midrule
HH    & $+2$   & $+1$   & $1/4$     \\
HT    & $0$    & $-1$   & $1/4$     \\
TH    & $0$    & $-1$   & $1/4$     \\
TT    & $-2$   & $+1$   & $1/4$     \\
\bottomrule
\end{tabular}
\end{center}

Notice that \textbf{HT and TH both give $(X=0, Y=-1)$}, so their probabilities add:
 \begin{itemize}
    \item $P(X = 0, Y = -1) = \frac{1}{4} + \frac{1}{4} = \frac{1}{2}$
    \item $P(X = +2, Y = +1) = \frac{1}{4}$
    \item $P(X = -2, Y = +1) = \frac{1}{4}$
\end{itemize}
All other $(x,y)$ pairs have probability 0.


We can now build the \textbf{joint probability distribution} $P_{X,Y}(x, y) = \mathbb{P}(X = x \text{ and } Y = y)$. After grouping outcomes that lead to the same 
$(X,Y)$ pair, the joint probability distribution of the random variables is:

\begin{center}
\begin{tabular}{ccc}
\toprule
X & Y & $P(X,Y)$ \\
\midrule
+2 & +1 & 1/4 \\
0  & -1 & 1/2 \\
-2 & +1 & 1/4 \\
\bottomrule
\end{tabular}
\end{center}
This will be useful later for calculating the conditional entropy later on.

\newpage

\section*{Entropy}

\subsection*{Information Content of a Single Outcome}

For a random variable $X$ with possible outcome $x$ and probability $P_X(x)$, the \textbf{information content} (or \emph{self-information}) of observing $x$ is defined as:
\[
I(x) = - \log_2 P_X(x)
\]

Intuitively:
\begin{itemize}
    \item Rare events (small $P_X(x)$) are more \textbf{surprising} and carry more information.  
    \item Common events (large $P_X(x)$) are less surprising and carry less information.  
\end{itemize}

\subsection*{Example 1: Fair Coin}

A fair coin has:
\[
P(\text{Heads}) = P(\text{Tails}) = 1/2
\]

\[
I(\text{Heads}) = I(\text{Tails}) = - \log_2(1/2) = 1 \text{ bit}
\]

Each flip of a fair coin provides 1 bit of information on average.

\subsection*{Example 2: Biased Coin}

Consider a biased coin with:
\[
P(\text{Heads}) = 0.9, \quad P(\text{Tails}) = 0.1
\]

Information content of each outcome:
\[
I(\text{Heads}) = - \log_2 0.9 \approx 0.152 \text{ bits}, \quad
I(\text{Tails}) = - \log_2 0.1 \approx 3.32 \text{ bits}
\]

Observing \textbf{Heads} is unsurprising and carries very little information, while observing \textbf{Tails} is rare and highly informative.

\subsection*{Entropy: Average Information}

The \textbf{entropy} of a random variable $X$ measures the \textbf{average information content} over all outcomes:
\[
H(X) = \mathbb{E}[I(X)] = - \sum_x P_X(x) \log_2 P_X(x)
\]

Entropy quantifies the \emph{expected surprise} or \emph{uncertainty} before observing the outcome.

\subsubsection*{Fair Coin}

\[
H(X) = - \left( \frac{1}{2} \log_2 \frac{1}{2} + \frac{1}{2} \log_2 \frac{1}{2} \right) = 1 \text{ bit}
\]

\subsubsection*{Biased Coin}

\[
H(X) = - \left(0.9 \log_2 0.9 + 0.1 \log_2 0.1 \right) \approx 0.61 \text{ bits}
\]

Because the biased coin outcome is more predictable, each flip provides less information on average than a fair coin.

\subsection*{Summary}

\begin{itemize}
    \item \textbf{Information} measures the surprise of a single outcome.  
    \item \textbf{Entropy} measures the \textbf{average information} across all outcomes.  
    \item High entropy = outcomes are uncertain (fair coin).  
    \item Low entropy = outcomes are predictable (biased coin).  
\end{itemize}








\section*{Additivity}

Because of the logarithm, entropy is additive. If $X$ and $Y$ are independent then:
\[
H(X, Y) = H(X) + H(Y)
\]
The unit of entropy is the \textbf{bit}.

If we flip $n$ independent fair coins $X_1, X_2, \dots, X_n$, the total entropy is additive:
\[
H(X_1, X_2, \dots, X_n) = H(X_1) + H(X_2) + \dots + H(X_n) = n \text{ bits}.
\]

Intuitively, additivity works because independent variables carry \emph{separate, non-overlapping information}. 
If the variables are \textbf{not independent}, knowing one gives partial information about the other, so the joint entropy is less than the sum:
\[
H(X, Y) < H(X) + H(Y) \quad \text{if $X$ and $Y$ are dependent.}
\]
We will see concrete examples of this later.


Back to our example, for $X$, which takes values $-2, 0, +2$ with probabilities $\frac{1}{4}, \frac{1}{2}, \frac{1}{4}$, we compute:
\[
\begin{aligned}
H(X) &= -\sum_x P_X(x) \log_2 P_X(x) \\
&= -\left[ \frac{1}{4} \log_2 \frac{1}{4} + \frac{1}{2} \log_2 \frac{1}{2} + \frac{1}{4} \log_2 \frac{1}{4} \right] \\
&= -\left[ \frac{1}{4}(-2) + \frac{1}{2}(-1) + \frac{1}{4}(-2) \right] = -[-1.5] = 1.5
\end{aligned}
\]
So \textbf{$H(X) = \frac{3}{2}$ bits}. 

Now let's calculate the entropy of $Y$

$Y$ can take either of the values $+1$ or $-1$, each with probability $\frac{1}{2}$, so:
\[
H(Y) = -\left[ \frac{1}{2} \log_2 \frac{1}{2} + \frac{1}{2} \log_2 \frac{1}{2} \right] = 1 \text{ bit}
\]
Knowing nothing, it takes 1 bit to describe whether the tosses matched.

\section*{Conditional Entropy $H(X \mid Y)$}

We are now working with \textbf{non-independent} random variables. Suppose we \textbf{first learn the value of $Y$}. How much uncertainty about $X$ remains?

First, we define the \textbf{conditional entropy for a specific value of $Y$}:

\[
H(X \mid Y = y) = - \sum_x P_{X \mid Y}(x \mid y) \, \log_2 P_{X \mid Y}(x \mid y),
\]

where the \textbf{conditional probability} is
\[
P_{X \mid Y}(x \mid y) = \frac{P_{X,Y}(x, y)}{P_Y(y)}, \quad \text{for } P_Y(y) > 0.
\]

\noindent In words: 

The conditional probability $P_{X \mid Y}(x \mid y)$ gives the probability that $X = x$ 
 given that $Y = y$. It is computed by dividing the joint probability $P_{X,Y}(x,y)$ 
by the marginal probability $P_Y(y)$.


The \textbf{conditional entropy of $X$ given $Y$} is then the \emph{average} of $H(X \mid Y = y)$ over all possible values of $Y$, weighted by their probabilities:

\[
H(X \mid Y) = \sum_y P_Y(y) \, H(X \mid Y = y).
\]

This expresses that the conditional entropy is the \textbf{average remaining uncertainty about $X$} once we know $Y$.








\newpage
We are working now with non-independent random variables. Suppose we \textbf{first learn the value of $Y$}. How much uncertainty about $X$ remains?

Let us first define:
\[
H(X \mid Y = y) = - \sum_x P_{X \mid Y}(x \mid y) \, \log_2 P_{X \mid Y}(x \mid y)
\]
where:
\[
P_{X \mid Y}(x \mid y) = \frac{P_{X,Y}(x, y)}{P_Y(y)}, \quad \text{for } P_Y(y) > 0
\]

\[\text{The conditional probability } P_{X \mid Y}(x \mid y) \text{ is the probability that } X = x 
\text{ given that } Y = y. \text{ It is calculated by dividing the joint probability } 
P_{X,Y}(x,y) \text{ by the marginal probability } P_Y(y).\]


We now have the definition:
\[
H(X \mid Y) = \sum_y P_Y(y) \cdot H(X \mid Y = y)
\]
This says that the conditional entropy 
is the average uncertainty remaining about 
$X$
once we know 
$Y$
averaged over all possible values that 
$Y$
can take, weighted by their probabilities."

\subsection*{When $Y = +1$ (tosses match $\to$ HH or TT):}
\begin{itemize}
    \item This happens with probability $P_Y(+1) = \frac{1}{2}$
    \item Given this, $X = +2$ (HH) or $X = -2$ (TT), each with conditional probability:
    \[
    P(X = +2 \mid Y = +1) = \frac{P(X=+2, Y=+1)}{P_Y(+1)} = \frac{1/4}{1/2} = \frac{1}{2}
    \]
    (same for $X = -2$)
    \item So the conditional entropy is:
    \[
    H(X \mid Y = +1) = -\left[ \frac{1}{2} \log_2 \frac{1}{2} + \frac{1}{2} \log_2 \frac{1}{2} \right] = 1 \text{ bit}
    \]
\end{itemize}
\subsection*{When $Y = -1$ (tosses differ $\to$ HT or TH):}
\begin{itemize}
    \item Probability $P_Y(-1) = \frac{1}{2}$
    \item Both HT and TH give $X = 0$, so:
    \[
    P(X = 0 \mid Y = -1) = 1
    \]
    \item Thus:
    \[
    H(X \mid Y = -1) = -[1 \cdot \log_2 1] = 0 \text{ bits}
    \]
\end{itemize}

\subsection*{Average over $Y$:}
\[
H(X \mid Y) = \frac{1}{2} \cdot 1 + \frac{1}{2} \cdot 0 = \frac{1}{2} \text{ bit}
\]
On average, once we know whether the tosses matched, we only need \textbf{half a bit more} to pin down $X$.

\section*{{Step 4} Mutual Information $I(X; Y)$}

\textbf{Mutual information} answers:  
``How many bits of uncertainty about $X$ are \textbf{removed} by knowing $Y$?''  
Or equivalently: ``How much do $X$ and $Y$ tell us about each other?''

Mathematically:
\[
I(X; Y) = H(X) - H(X \mid Y)
\]

Plugging in:
\[
I(X; Y) = \frac{3}{2} - \frac{1}{2} = 1 \text{ bit}
\]

We can \textbf{verify this another way}: check $H(Y \mid X)$.

\begin{itemize}
    \item If $X = 0$ $\to$ must be HT or TH $\to$ $Y = -1$ for sure
    \item If $X = +2$ $\to$ must be HH $\to$ $Y = +1$
    \item If $X = -2$ $\to$ must be TT $\to$ $Y = +1$
\end{itemize}
So \textbf{knowing $X$ tells us $Y$ exactly} $\to$ $H(Y \mid X) = 0$

Thus:
\[
I(X; Y) = H(Y) - H(Y \mid X) = 1 - 0 = 1 \text{ bit}
\]

Both ways agree.

\textbf{Interpretation}: $X$ and $Y$ share \textbf{1 bit of information}.  
In fact, $Y$ is a \textbf{function of $X$} (you can compute matching status from the head--tail difference), so all of $Y$'s information is contained in $X$. But $X$ has extra detail (sign of the difference) that $Y$ doesn't capture—which is why $H(X) > H(Y)$.

\section*{{Final Results}}

\begin{center}
\begin{tabular}{lc}
\toprule
Quantity & Value \\
\midrule
$H(X)$ & $1.5$ bits \\
$H(Y)$ & $1$ bit \\
$H(X \mid Y)$ & $0.5$ bits \\
$H(Y \mid X)$ & $0$ bits \\
$I(X; Y)$ & $1$ bit \\
\bottomrule
\end{tabular}
\end{center}

\section*{{Key Takeaway}}

Even though we started with \textbf{four equally likely physical outcomes} (HH, HT, TH, TT), the random variables $X$ and $Y$ \textbf{compress} this information in different ways.
 \begin{itemize}
    \item $X$ distinguishes \textbf{three cases} (++ , +--/--+ , --)
    \item $Y$ distinguishes \textbf{two cases} (same vs. different)
Their \textbf{mutual information} quantifies exactly how much these summaries overlap—and in this case, \textbf{all of $Y$'s content is inside $X$}, but not vice versa.
\end{itemize}

This is the power of information theory: it lets us \textbf{measure shared knowledge} between any two descriptions of the same experiment—whether they’re numbers, categories, or even words.









\section*{\texorpdfstring{\faLightbulb}{Intuition} What Is Conditional Entropy?}

\textbf{Conditional entropy} \(H(X \mid Y)\) quantifies the \textbf{average remaining uncertainty} in a random variable \(X\) after the value of another random variable \(Y\) has been observed. In other words, it answers:

\begin{quote}
``If I know \(Y\), how much don’t I still know about \(X\)—on average?''
\end{quote}

It is always true that \(H(X \mid Y) \leq H(X)\), with equality if and only if \(X\) and \(Y\) are statistically independent.

\section*{\texorpdfstring{\faKey}{Key Building Block} Joint Probability}

Before defining conditional entropy, we need the \textbf{joint probability distribution} of \(X\) and \(Y\):

\[
\boxed{P_{X,Y}(x, y) = \mathbb{P}(X = x \text{ and } Y = y)}
\]

This is the probability that \textbf{both} \(X\) takes the value \(x\) \textbf{and} \(Y\) takes the value \(y\) in the same trial of the experiment. The joint distribution fully describes how \(X\) and \(Y\) behave together.

From it, we obtain the \textbf{marginal distributions} by summing out the other variable:
\[
P_X(x) = \sum_{y} P_{X,Y}(x, y), \qquad
P_Y(y) = \sum_{x} P_{X,Y}(x, y)
\]

\section*{\texorpdfstring{\faRuler}{Three Equivalent Expressions} for Conditional Entropy}

All three formulas below compute the \textbf{same quantity}—they are just different ways of writing the same idea.

\subsection*{1. Average of Conditional Entropies (Conceptual Form)}

\[
\boxed{
H(X \mid Y) = \sum_{y} P_Y(y) \, H(X \mid Y = y)
}
\]

\begin{itemize}
    \item Here, \(H(X \mid Y = y)\) is the entropy of \(X\) \textbf{in the scenario where \(Y = y\) is known}.
    \item This form emphasizes that conditional entropy is an \textbf{expectation over the possible values of \(Y\)}.
\end{itemize}

\subsection*{2. Expanded with Conditional Probabilities}

\[
\boxed{
H(X \mid Y) = -\sum_{y} P_Y(y) \sum_{x} P_{X \mid Y}(x \mid y) \log_2 P_{X \mid Y}(x \mid y)
}
\]

\begin{itemize}
    \item The \textbf{conditional probability} is defined (when \(P_Y(y) > 0\)) as:
    \[
    P_{X \mid Y}(x \mid y) = \frac{P_{X,Y}(x, y)}{P_Y(y)}
    \]
    \item This expression simply substitutes the definition of entropy into the average from Form 1.
    \item It highlights that for each fixed \(y\), we compute the entropy of the \textbf{conditional distribution} \(P_{X \mid Y}(\cdot \mid y)\).
\end{itemize}


\subsection*{3. Joint-Probability Form (Computational Form)}

\[
\boxed{
H(X \mid Y) = -\sum_{x} \sum_{y} P_{X,Y}(x, y) \log_2 \left( \frac{P_{X,Y}(x, y)}{P_Y(y)} \right)
}
\]

\begin{itemize}
    \item This version uses \textbf{only the joint distribution \(P_{X,Y}(x,y)\)} and the marginal \(P_Y(y)\)—no explicit reference to conditional probabilities is needed in the notation.
    \item It is obtained by replacing \(P_Y(y) P_{X \mid Y}(x \mid y)\) with \(P_{X,Y}(x, y)\) in Form 2.
    \item This is often the most convenient form when working from a joint probability table or empirical data.
\end{itemize}

\section*{\texorpdfstring{\faComment}{Interpretation Summary}}

\begin{itemize}
    \item \textbf{Form 1} tells you \textit{what conditional entropy means}: an average of uncertainties.
    \item \textbf{Form 2} shows \textit{how it’s built}: using conditional probability distributions.
    \item \textbf{Form 3} shows \textit{how to compute it}: directly from joint and marginal probabilities.
\end{itemize}

All three are mathematically identical and interchangeable—choose the one that best fits your context: understanding, derivation, or calculation.

\begin{quote}
\textbf{Remember}: Conditional entropy measures \textbf{residual uncertainty}, not shared information. The shared information is captured by \textbf{mutual information}:
\[
I(X;Y) = H(X) - H(X \mid Y)
\]
\end{quote}




\section*{Mutual Information and Bayes' Theorem: \\ Information Gain from Bayesian Updating}

Mutual information and Bayes' theorem are deeply connected:
\begin{itemize}
\item \textbf{Bayes' theorem} tells you how to update a \emph{single prior} \(P(X)\) to a \emph{posterior} \(P(X \mid Y = y)\) after observing a specific outcome \(Y = y\).
    \item \textbf{Mutual information} \(I(X;Y)\) tells you the \emph{expected reduction in uncertainty} about \(X\) \emph{on average} over all possible outcomes \(y\), weighted by their likelihood.
\end{itemize}


In short:
\begin{center}
    \textbf{Mutual information = Expected information gain from Bayesian updating.}
\end{center}

\section*{Formal Connection}

\subsection*{1. Bayes' Theorem and Information Gain for a Single Observation}

Given a prior distribution \(P(X)\), observing \(Y = y\) yields the posterior via Bayes' rule:
\[
P(X \mid Y = y) = \frac{P(Y = y \mid X)\, P(X)}{P(Y = y)}.
\]

The \textbf{information gained} from this observation is measured by the Kullback–Leibler (KL) divergence from the prior to the posterior:
\[
D_{\mathrm{KL}}\big(P(X \mid Y = y) \,\|\, P(X)\big) 
= \sum_{x} P(x \mid y) \log_2 \frac{P(x \mid y)}{P(x)}.
\]
This quantifies how much the belief about \(X\) changed due to seeing \(Y = y\).

\subsection*{2. Mutual Information as Expected KL Divergence}

Mutual information is the expectation of this KL divergence over all possible outcomes \(y\):
\[
\boxed{
I(X;Y) = \mathbb{E}_{Y}\!\left[ D_{\mathrm{KL}}\big(P(X \mid Y) \,\|\, P(X)\big) \right]
= \sum_{y} P(y) \sum_{x} P(x \mid y) \log_2 \frac{P(x \mid y)}{P(x)}.
}
\]

Using \(P(x,y) = P(x \mid y) P(y)\), this is algebraically equivalent to the standard definition:
\[
I(X;Y) = \sum_{x,y} P(x,y) \log_2 \left( \frac{P(x,y)}{P(x) P(y)} \right).
\]

Thus, mutual information is precisely the \textbf{average information gain} from applying Bayes' rule.

\section*{Interpretation}


[leftmargin=*]\begin{itemize}
    \item \textbf{KL divergence} \(D_{\mathrm{KL}}(P(X|y) \| P(X))\): \\
    ``How many extra bits would I waste if I encoded \(X\) using the prior instead of the correct posterior?'' \\
    -> This is the \emph{information gained} from observing \(Y = y\).

    \item \textbf{Mutual information} \(I(X;Y)\): \\
    ``On average, how many bits do I gain about \(X\) per observation of \(Y\)?'' \\
    -> This is the \emph{expected information gain} from Bayesian updating.
\end{itemize}

\section*{Concrete Dice Example: Connecting Bayes + Mutual Information}

Recall the setup:
\begin{itemize}
\item Roll two fair dice: \(D_1, D_2\).
    \item Let \(X = D_1\) (first die), \(Y = D_1 + D_2\) (sum).
    \item Prior: \(P(X = x) = \frac{1}{6}\) for \(x = 1,\dots,6\).
\end{itemize}

\subsection*{Bayesian Updating for Specific Observations}

\begin{enumerate}
    \item \textbf{Observe \(Y = 2\)}: \\
    Only possible if \(D_1 = 1, D_2 = 1\). \\
    Posterior: \(P(X = 1 \mid Y = 2) = 1\), others 0. \\
    Information gain:
    \[
    D_{\mathrm{KL}}(P(X|2) \| P(X)) = \log_2 \frac{1}{1/6} = \log_2 6 \approx 2.585 \text{ bits}.
    \]

    \item \textbf{Observe \(Y = 7\)}: \\
    All pairs \((1,6), (2,5), \dots, (6,1)\) equally likely. \\
    Posterior: \(P(X = x \mid Y = 7) = \frac{1}{6}\) for all \(x\). \\
    Information gain:
    \[
    D_{\mathrm{KL}}(P(X|7) \| P(X)) = \sum_{x=1}^6 \frac{1}{6} \log_2 \frac{1/6}{1/6} = 0 \text{ bits}.
    \]
\end{enumerate}

\subsection*{Mutual Information as the Average Gain}

Mutual information averages these gains over all possible sums:
\[
I(X;Y) = \sum_{y=2}^{12} P(Y = y) \cdot D_{\mathrm{KL}}\big(P(X \mid Y = y) \,\|\, P(X)\big).
\]

Using the known distribution of \(Y\):
\begin{center}
\begin{tabular}{c|ccccccccccc}
\(y\) & 2 & 3 & 4 & 5 & 6 & 7 & 8 & 9 & 10 & 11 & 12 \\
\midrule
\(P(Y=y)\) & \(\frac{1}{36}\) & \(\frac{2}{36}\) & \(\frac{3}{36}\) & \(\frac{4}{36}\) & \(\frac{5}{36}\) & \(\frac{6}{36}\) & \(\frac{5}{36}\) & \(\frac{4}{36}\) & \(\frac{3}{36}\) & \(\frac{2}{36}\) & \(\frac{1}{36}\) \\
\(n_y\) (support size) & 1 & 2 & 3 & 4 & 5 & 6 & 5 & 4 & 3 & 2 & 1 \\
\(D_{\mathrm{KL}}\) & \(\log_2 6\) & \(\log_2 3\) & \(\log_2 2\) & \(\log_2 \tfrac{4}{3}\)? & \dots & 0 & \dots & & & &
\end{tabular}
\end{center}

More precisely, since \(P(X \mid Y = y)\) is uniform over \(n_y\) values,
\[
D_{\mathrm{KL}}\big(P(X \mid Y = y) \,\|\, P(X)\big) = \log_2 n_y - \log_2 6 = \log_2 \left( \frac{n_y}{6} \right) \quad \text{(but note: actually } = \log_2 n_y \text{ because prior entropy cancels in expectation)}.
\]

Carrying out the full calculation (as in the earlier example) yields:
\[
I(X;Y) \approx 0.6895 \text{ bits}.
\]

This is exactly the \textbf{expected information gain} from observing the sum and applying Bayes' rule.

\section*{Why This Matters}

[leftmargin=*]\begin{itemize}
    \item \textbf{Bayesian inference}: Every Bayes update provides information; mutual information quantifies its average value.
    \item \textbf{Experimental design}: Choose observations \(Y\) that maximize \(I(X;Y)\) to learn most about \(X\).
    \item \textbf{Machine learning}: In variational inference, mutual information appears in bounds on model evidence.
    \item \textbf{Cognitive science}: Models perception as Bayesian updating, with mutual information measuring sensory informativeness.
\end{itemize}

\section*{Summary}

\begin{center}
\begin{tabular}{ll}
\toprule
Concept & Role \\
\midrule
Bayes' theorem & Updates prior \(P(X)\) → posterior \(P(X \mid Y = y)\) for a \emph{specific} \(y\). \\
KL divergence \(D_{\mathrm{KL}}(P(X|y) \| P(X))\) & Information gained from that \emph{specific} update. \\
Mutual information \(I(X;Y)\) & \emph{Expected} information gain over all \(y\). \\
\bottomrule
\end{tabular}
\end{center}

Thus, mutual information provides an \textbf{information-theoretic foundation for Bayesian learning}: it measures how much an observation is expected to teach us about the world.





\end{document}













\end{document}















\end{document}
