\documentclass[12pt]{article}
\usepackage{amsmath}
\usepackage{geometry}
\geometry{margin=1in}

\title{Independent Events in Two-Dice Experiments}
\author{}
\date{}

\begin{document}
\maketitle

\Large

Two events are said to be \emph{independent} if:
\[ P(A \cap B) = P(A)P(B)\]

Mutually exclusive events are never independent. In this section we look at experiments involving the rolling of two dice and give two examples of independent events and two examples of dependent events.

The key point is that when events or \emph{not} independent the conditional probability doesn't equal the ``raw'' probability:
\[ P(A | B) \neq P(A) \]
In other words, knowing that $B$ occured gives you some predictive knowledge over whether $A$ will occur.


\section*{Example 1: Sum Equals 7 and First Die is 6 [Independent]}

Consider rolling two fair six-sided dice. Define the events:

\[
A = \{\text{sum of the dice is 7}\} = \{(1,6),(2,5),(3,4),(4,3),(5,2),(6,1)\}
\]

\[
B = \{\text{first die shows 6}\} = \{(6,1),(6,2),(6,3),(6,4),(6,5),(6,6)\}
\]

\subsection*{Step 1: Probabilities}

\[
P(A) = \frac{6}{36} = \frac{1}{6}, \quad
P(B) = \frac{6}{36} = \frac{1}{6}
\]

\[
A \cap B = \{(6,1)\} \implies P(A \cap B) = \frac{1}{36}
\]

\subsection*{Step 2: Check Independence}

Two events are independent if:

\[
P(A \cap B) = P(A)P(B)
\]

Compute:

\[
P(A)P(B) = \frac{1}{6} \cdot \frac{1}{6} = \frac{1}{36} = P(A \cap B)
\]

Thus, \(A\) and \(B\) are \textbf{independent}.

\subsection*{Step 3: Intuition}

Knowing that the first die is 6 does not change the probability that the sum of the dice is 7.  
- Original probability: \(P(A) = 1/6\)  
- Conditional probability: \(P(A \mid B) = \frac{P(A \cap B)}{P(B)} = \frac{1/36}{6/36} = 1/6\)  

Since \(P(A \mid B) = P(A)\), the events are independent.

\bigskip

\section*{Example 2: Sum is Even and First Die is Even [Independent]}

Consider rolling two fair six-sided dice. Define the events:

\[
A = \{\text{sum of the dice is even}\}, \quad
B = \{\text{first die shows even}\}
\]

\subsection*{Step 1: Probabilities}

There are 36 equally likely outcomes. Half of them have an even sum:

\[
P(A) = \frac{18}{36} = 0.5
\]

Half of the outcomes have an even first die:

\[
P(B) = \frac{18}{36} = 0.5
\]

The intersection is outcomes where first die is even and sum is even:

\[
A \cap B = \{(2,2),(2,4),(2,6),(4,2),(4,4),(4,6),(6,2),(6,4),(6,6)\}
\]

\[
P(A \cap B) = \frac{9}{36} = 0.25
\]

\subsection*{Step 2: Check Independence}

\[
P(A)P(B) = 0.5 \cdot 0.5 = 0.25 = P(A \cap B)
\]

So \(A\) and \(B\) are \textbf{independent}.

\subsection*{Step 3: Intuition}

- Event \(A\) depends on the parity of the sum.  
- Event \(B\) fixes the parity of the first die.  
- Whether the sum is even depends only on whether the two dice have the same parity.  
- If the first die is even, the sum is even if the second die is even (50\% chance).  
- If the first die is odd, the sum is even if the second die is odd (50\% chance).  

In both cases, \(P(A \mid B) = P(A) = 0.5\).  
Knowing that the first die is even does not change the probability that the sum is even.

\section*{Example 3: Sum Even and Sum Greater than 7 [Dependent]}

Consider rolling two fair six-sided dice. Define the events:

\[
A = \{\text{sum of the dice is even}\}, \quad
B = \{\text{sum of the dice is greater than 7}\}
\]

\subsection*{Step 1: Probabilities}

- There are 36 equally likely outcomes in total.  
- Number of outcomes with an even sum: 18  
\[
P(A) = \frac{18}{36} = 0.5
\]

- Number of outcomes with sum greater than 7: 15  
\[
P(B) = \frac{15}{36} \approx 0.4167
\]

- Intersection (sum even \textbf{and} sum $> 7$): sums 8, 10, 12 → 9 outcomes  
\[
P(A \cap B) = \frac{9}{36} = 0.25
\]

\subsection*{Step 2: Check Independence}

Two events are independent if:

\[
P(A \cap B) = P(A)P(B)
\]

Compute:

\[
P(A)P(B) = 0.5 \cdot 0.4167 = 0.2083 \neq 0.25
\]

\textbf{Conclusion:} \(A\) and \(B\) are \textbf{not independent}.

\subsection*{Step 3: Conditional Probability and Intuition}

The conditional probability of \(A\) given \(B\) is:

\[
P(A \mid B) = \frac{P(A \cap B)}{P(B)} = \frac{0.25}{0.4167} \approx 0.6
\]

- Original probability of sum even: \(P(A) = 0.5\)  
- Probability of sum even given sum $>7$: \(P(A \mid B) \approx 0.6\)

\bigskip

\textbf{Intuition:}  

- Event \(A\) depends on the parity of the sum.  
- Event \(B\) restricts outcomes to sums greater than 7.  
- Among sums $>7$ (8, 9, 10, 11, 12), there are proportionally more even sums (8, 10, 12) than odd sums (9, 11).  
- Therefore, knowing that \(B\) occurred increases the probability of \(A\), which is exactly why \(A\) and \(B\) are \textbf{dependent}.


\section*{Example 4: At Least One 6 and Sum Greater Than 10 [Dependent]}

Consider rolling two fair six-sided dice. Define the events:

\[
C = \{\text{at least one die shows 6}\}, \quad
D = \{\text{sum of the dice is greater than 10}\}
\]

\subsection*{Probabilities}

- Total outcomes: 36  
- Outcomes with at least one 6: 11  
\[
P(C) = \frac{11}{36} \approx 0.3056
\]  

- Outcomes with sum $> 10$: 3  
\[
P(D) = \frac{3}{36} \approx 0.0833
\]  

- Intersection (at least one 6 and sum $>10$): 3 outcomes  
\[
P(C \cap D) = \frac{3}{36} = 0.0833
\]

\subsection*{Check Independence}

Two events are independent if \(P(C \cap D) = P(C) P(D)\).  

\[
P(C)P(D) = 0.3056 \cdot 0.0833 \approx 0.0254 \neq 0.0833
\]

\textbf{Conclusion:} \(C\) and \(D\) are \textbf{not independent}.

\subsection*{Conditional Probability and Intuition}

\[
P(D \mid C) = \frac{P(C \cap D)}{P(C)} = \frac{0.0833}{0.3056} \approx 0.2727
\]

- Original probability of sum $>10$: \(P(D) = 0.0833\)  
- Probability of sum $>10$ given at least one die is 6: \(P(D \mid C) \approx 0.2727\)

\textbf{Intuition:}  

- Event \(D\) requires a high sum ($>10$).  
- Event \(C\) ensures at least one die is 6.  
- High sums are more likely if at least one die is 6, so knowing \(C\) occurred significantly increases the probability of \(D\), showing dependence.





\end{document}

