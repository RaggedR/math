\documentclass{article}
\usepackage{amsmath, amssymb}
\usepackage{enumitem}

\title{Orthogonal Projection Matrices and Symmetry}
\author{}
\date{}

\begin{document}

\maketitle

\Large

Let $ P \in \mathbb{R}^{n \times n} $ be a projection matrix, meaning $ P^2 = P $. We say $ P $ is an \textbf{orthogonal projection} if it projects vectors orthogonally onto its column space. We will use the theory of the \textbf{four fundamental subspaces} of a matrix to prove the following key result:

\begin{center}
    \textbf{A projection matrix $ P $ is orthogonal if and only if $ P^T = P $.}
\end{center}

Recall the four fundamental subspaces of a matrix $ P $:
\begin{itemize}
\item Column space: $ \mathrm{Col}(P) $
    \item Null space: $ \mathrm{Nul}(P) $
    \item Row space: $ \mathrm{Row}(P) $
    \item Left null space: $ \mathrm{Nul}(P^T) $
\end{itemize}

For any matrix, we have the orthogonal decompositions:
\[
\mathbb{R}^n = \mathrm{Col}(P) \oplus \mathrm{Nul}(P^T), \quad 
\mathbb{R}^n = \mathrm{Row}(P) \oplus \mathrm{Nul}(P)
\]
and since $ \mathrm{Row}(P) = \mathrm{Col}(P^T) $, the symmetry condition $ P^T = P $ implies $ \mathrm{Row}(P) = \mathrm{Col}(P) $.

Now, suppose $ P $ is a projection, so $ P^2 = P $. This means that for any $ \mathbf{x} \in \mathbb{R}^n $, we can write:
\[
\mathbf{x} = P\mathbf{x} + (\mathbf{x} - P\mathbf{x})
\]
where $ P\mathbf{x} \in \mathrm{Col}(P) $ and $ \mathbf{x} - P\mathbf{x} \in \mathrm{Nul}(P) $, because:
\[
P(\mathbf{x} - P\mathbf{x}) = P\mathbf{x} - P^2\mathbf{x} = P\mathbf{x} - P\mathbf{x} = \mathbf{0}.
\]

Thus, $ \mathbb{R}^n = \mathrm{Col}(P) \oplus \mathrm{Nul}(P) $, and $ P $ projects $ \mathbf{x} $ onto $ \mathrm{Col}(P) $ along $ \mathrm{Nul}(P) $.

\bigskip

\noindent \textbf{Definition:} The projection $ P $ is \textbf{orthogonal} if $ \mathrm{Nul}(P) \perp \mathrm{Col}(P) $. That is, the projection direction is orthogonal to the range.

We now prove the main result in two parts.

\bigskip

\noindent \textbf{($ \Rightarrow $)} Suppose $ P $ is an orthogonal projection. Then $ \mathrm{Nul}(P) \perp \mathrm{Col}(P) $.

Since $ \mathbb{R}^n = \mathrm{Col}(P) \oplus \mathrm{Nul}(P) $ and the sum is orthogonal, we have:
\[
\mathrm{Nul}(P) = \mathrm{Col}(P)^\perp.
\]

But from the fundamental theorem of linear algebra,
\[
\mathrm{Col}(P)^\perp = \mathrm{Nul}(P^T).
\]

Therefore,
\[
\mathrm{Nul}(P) = \mathrm{Nul}(P^T).
\]

Also, since $ P $ is a projection, $ \mathrm{Col}(P) $ consists of all vectors $ \mathbf{y} $ such that $ P\mathbf{y} = \mathbf{y} $. Similarly, $ \mathrm{Col}(P^T) $ contains vectors fixed by $ P^T $ if $ P^T $ is also idempotent, but more importantly, we use subspace equality.

We already have $ \mathrm{Nul}(P) = \mathrm{Nul}(P^T) $. Taking orthogonal complements:
\[
\mathrm{Nul}(P)^\perp = \mathrm{Nul}(P^T)^\perp \implies \mathrm{Col}(P^T) = \mathrm{Col}(P),
\]
since $ \mathrm{Col}(P) = \mathrm{Nul}(P)^\perp $ and $ \mathrm{Col}(P^T) = \mathrm{Nul}(P^T)^\perp $.

So $ \mathrm{Col}(P) = \mathrm{Col}(P^T) $ and $ \mathrm{Nul}(P) = \mathrm{Nul}(P^T) $.

Now, consider that both $ P $ and $ P^T $ are projections (note: $ (P^T)^2 = (P^2)^T = P^T $, so $ P^T $ is also idempotent). Both project onto the same column space $ \mathrm{Col}(P) $, and along the same null space $ \mathrm{Nul}(P) $. Since a projection is uniquely determined by its range and null space (as complementary subspaces), and here both $ P $ and $ P^T $ have:
\begin{itemize}
\item Range: $ \mathrm{Col}(P) $
    \item Null space: $ \mathrm{Nul}(P) $
it follows that $ P = P^T $.
Hence, $ P^T = P $.
\end{itemize}

\bigskip

\noindent \textbf{($ \Leftarrow $)} Conversely, suppose $ P^T = P $ and $ P^2 = P $. We show $ P $ is an orthogonal projection.

Since $ P $ is symmetric, $ \mathrm{Col}(P) = \mathrm{Row}(P) $, and from the fundamental theorem:
\[
\mathrm{Nul}(P) = \mathrm{Row}(P)^\perp = \mathrm{Col}(P)^\perp.
\]

Thus, $ \mathrm{Nul}(P) \perp \mathrm{Col}(P) $, which means the projection is along the orthogonal complement of the column space. Therefore, $ P $ is an orthogonal projection.

\bigskip

\noindent \textbf{Conclusion:} A projection matrix $ P $ (i.e., $ P^2 = P $) is an orthogonal projection if and only if $ P^T = P $.

\end{document}
