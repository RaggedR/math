\chapter{The Dual Space}

\Large

\section{The Dual Space}

Hopefully the reader has seen the idea of the \emph{dual space} before, otherwise this section will probably be a little confusing. 

Let $ V $ be a finite dimensional vector space over the real numbers $ \mathbb{R} $  
A \textbf{linear functional} is a function $ f: V \to \mathbb{R} $ such that for all vectors $ \mathbf{u}, \mathbf{v} \in V $ and scalars $ \alpha, \beta \in \mathbb{R} $,
\[
f(\alpha \mathbf{u} + \beta \mathbf{v}) = \alpha f(\mathbf{u}) + \beta f(\mathbf{v}).
\]



The \textbf{dual space} of $ V $, denoted $ V^* $, is the set of all \textbf{linear functionals} on $ V $. That is,
\[
V^* = \{ f: V \to \mathbb{R} \mid f \text{ is linear} \}.
\]



The dual space $ V^* $ is itself a vector space over $ \mathbb{R} $, with vector addition and scalar multiplication defined \textit{pointwise}:
\begin{align*}
(f + g)(\mathbf{v}) &= f(\mathbf{v}) + g(\mathbf{v}), \\
(\alpha f)(\mathbf{v}) &= \alpha \, f(\mathbf{v}),
\end{align*}
for all $ f, g \in V^* $, $ \alpha \in \mathbb{R} $, and $ \mathbf{v} \in V $.



\subsection*{\Large Key Point}

One of the most important properties of the dual space of $V$ is that the standard dot product allows us to associate every vector $\mathbf{v} \in V$ with a linear functional $f_{\mathbf{v}} \in V^*$ via the rule

\vspace{1em}
\[
\boxed{f_{\mathbf{v}}(\mathbf{y}) = \mathbf{v} \cdot \mathbf{y} = \mathbf{v}^\top \mathbf{y}, 
\quad \text{for all } \mathbf{y} \in V}.
\]
\vspace{1em}

\begin{theorem}[Finite-Dimensional Riesz Representation Theorem]\label{isomorphism}
Let $V = \mathbb{R}^n$ with the standard inner product. 
Then the map
\[
\psi : V \to V^*, \quad 
\psi(\mathbf{v})(\mathbf{y}) = \mathbf{v}^\top \mathbf{y},
\]
is a vector space isomorphism.
\begin{proof}
First, linearity: for all $\alpha,\beta \in \mathbb{R}$ and $\mathbf{v},\mathbf{w},\mathbf{y} \in V$,
\[
f_{\alpha \mathbf{v} + \beta \mathbf{w}}(\mathbf{y}) 
= (\alpha \mathbf{v} + \beta \mathbf{w})^\top \mathbf{y} 
= \alpha \mathbf{v}^\top \mathbf{y} + \beta \mathbf{w}^\top \mathbf{y} 
= \alpha f_{\mathbf{v}}(\mathbf{y}) + \beta f_{\mathbf{w}}(\mathbf{y}).
\]

Next, injectivity: if $f_{\mathbf{v}} = 0$, then $\mathbf{v}^\top \mathbf{y} = 0$ for all $\mathbf{y}$.  
In particular, $\mathbf{v}^\top \mathbf{v} = \|\mathbf{v}\|^2 = 0$, so $\mathbf{v} = \mathbf{0}$.  

Finally, surjectivity: given $g \in V^*$, define $v_i = g(\mathbf{e}_i)$ for the standard basis $\{\mathbf{e}_i\}$.  
Set $\mathbf{v} = \sum_i v_i \mathbf{e}_i$. Then for each basis vector,
\[
f_{\mathbf{v}}(\mathbf{e}_i) = \mathbf{v}^\top \mathbf{e}_i = v_i = g(\mathbf{e}_i).
\]
Since $f_{\mathbf{v}}$ and $g$ agree on a basis, they are equal.  

Thus $\psi$ is linear, injective, and surjective, hence an isomorphism.
\end{proof}
\end{theorem}
\begin{corollary}
$\dim V^* = \dim V$ for finite-dimensional $V$
\end{corollary}

As a consequence of the above, suppose that $ \{ \mathbf{e}_1, \dots, \mathbf{e}_n \} $ is a basis for $ V $, then there is a uniquely associated \textbf{dual basis} $ \{ \mathbf{e}^1, \dots, \mathbf{e}^n \} \subseteq V^* $ defined by
    \[
    \mathbf{e}^i(\mathbf{e}_j) = \delta^i_j =
    \begin{cases}
    1 & \text{if } i = j, \\
    0 & \text{if } i \ne j.
    \end{cases}
    \]
    
    Every linear functional $ f \in V^* $ can be uniquely expressed as
    \[
    f = \sum_{i=1}^n f(\mathbf{e}_i) \, \mathbf{e}^i.
    \]

\section*{\Large Some intuition}

It can sometimes help to think of vectors in $V$ as column vectors and vectors in $V^*$ as row vectors. The dot product is now the application of a linear functional to a vector. The transpose is the isomoprhism between $V$ and $V^*$. 



















\section*{\huge The Adjoint}

The reason why we have gone to all the trouble of introducing the dual space is that the transpose of a matrix is the natural matrix representation of the \textbf{adjoint} of a linear map. 

\subsection*{\Large The Adjoint of a Linear Map}

Let $ T: V \to W $ be a linear map between finite-dimensional real vector spaces.  
The \textbf{dual map} $ T^*: W^* \to V^* $ is defined by
\[
(T^* f)(\mathbf{v}) = f(T\mathbf{v}) \quad \text{for all } f \in W^*,\ \mathbf{v} \in V.
\]

In words: to evaluate $ T^* f $ at a vector $ \mathbf{v} $, first apply $ T $ to $ \mathbf{v} $, then apply the functional $ f $ to the result.  

\begin{theorem}
The matrix of the dual map $ T^* $ is the transpose of the matrix of $ T $.
\begin{proof}
Choose bases:

\begin{itemize}
\item $ \mathcal{B} = \{ \mathbf{v}_1, \dots, \mathbf{v}_n \} $ for $ V $,
    \item $ \mathcal{C} = \{ \mathbf{w}_1, \dots, \mathbf{w}_m \} $ for $ W $,
    \item $ \mathcal{B}^* = \{ \mathbf{v}^1, \dots, \mathbf{v}^n \} $ and $ \mathcal{C}^* = \{ \mathbf{w}^1, \dots, \mathbf{w}^m \} $ for the dual bases.
\end{itemize}


Suppose the matrix of $ T $ with respect to $ \mathcal{B}, \mathcal{C} $ is $ A = (a_{ij})$, then the image of $v_1$ is the first column of $A$:
\[
T(\mathbf{v}_1) = \sum_{i=1}^{m} A_{i1} \, \mathbf{w}_i
\]

More generally, the image of $v_j$ is the $j$th column of $A$:
\[
T(\mathbf{v}_j) = \sum_{i=1}^m a_{ij} \mathbf{w}_i.
\]

We compute the matrix of $ T^* $ with respect to $ \mathcal{C}^*, \mathcal{B}^* $.  
For any $ k \in \{1, \dots, m\} $ and $ j \in \{1, \dots, n\} $,
\[
(T^* \mathbf{w}^k)(\mathbf{v}_j) = \mathbf{w}^k(T \mathbf{v}_j) = \mathbf{w}^k\left( \sum_{i=1}^m a_{ij} \mathbf{w}_i \right) = a_{kj}.
\]

On the other hand, if the matrix of $ T^* $ is $ B = (b_{\ell k}) $, then
\[
T^*(\mathbf{w}^1) = \sum_{\ell=1}^{n} b_{\ell 1} \, \mathbf{v}^\ell
\]


More generally:
\[
T^*(\mathbf{w}^k) = \sum_{\ell=1}^n b_{\ell k} \mathbf{v}^\ell,
\]
so
\[
(T^* \mathbf{w}^k)(\mathbf{v}_j) = b_{j k}.
\]

Comparing both expressions gives $ b_{j k} = a_{k j} $, so $ B = A^\top $.


That is the matrix of the dual map $ T^* $ is the transpose of the matrix of $ T $.
\end{proof}
\end{theorem}










\subsection{\Large The Four Fundamental Subspaces}

Recall from Chapter 1 that
the key orthogonal decompositions are:
\[
\mathbb{R}^n = \operatorname{Row}(A) \oplus \operatorname{Nul}(A), \quad
\mathbb{R}^m = \operatorname{Col}(A) \oplus \operatorname{Nul}(A^\top),
\]
or equivalently:
\[
\operatorname{Nul}(A) = \operatorname{Row}(A)^\perp, \quad
\operatorname{Nul}(A^\top) = \operatorname{Col}(A)^\perp.
\]

\section*{\Large The Annihilator}

Let $ V $ be a finite-dimensional vector space, and let $ W \subseteq V $ be a subspace.  
The \textbf{annihilator} of $ W $ is:
\[
W^\circ = \{ f \in V^* : f(\mathbf{w}) = 0 \text{ for all } \mathbf{w} \in W \},
\]
where $ V^* $ is the \textbf{dual space} (the space of linear functionals on $ V $).


\subsection{\Large Some notation}

If one is not careful then it is easy to confuse the vector space $V$ with the vector spacd $V^*$. For this reason we will use the following notation. Suppose that $A \subseteq V$ and $B \subseteq V^*R$ then we write:

\[ A \cong B \]

When we mean to say that: 

\[A = \psi^{-1} [B]\]

where $\psi$ is the isomorphism defined in Theorem \ref{isomorphism}. We come now to the most important result in this Chapter.

\begin{theorem} 
\[
\nul(A^\top) \cong (\col(A))^\circ, \qquad
\nul(A) \cong (\row(A))^\circ,
\]
where the annihilator is taken inside the appropriate dual space, and the isomorphism is the one induced by the dot product.
\begin{proof}
\medskip

We only need to prove one of these results, and the second one follows by symmetry.

Let $A$ be an $m \times n$ real matrix. We will prove that, under the natural identification of vectors in $\R^m$ with linear functionals on $\R^m$ provided by the dot product, the null space of $A^\top$ corresponds, under isomorphism, to the annihilator of the column space of $A$. That is,
\[
\nul(A^\top) \cong (\col(A))^\circ.
\]

We proceed step by step, explaining the meaning of each concept and how they relate.


The \textbf{column space} of $A$ is the subspace of $\R^m$ defined by
\[
\col(A) = \{ A\mathbf{x} \mid \mathbf{x} \in \R^n \} \subseteq \R^m.
\]
Its \textbf{annihilator} is the set of all linear functionals on $\R^m$ that vanish on every vector in $\col(A)$:
\[
(\col(A))^\circ = \left\{ f \in (\R^m)^* \,\middle|\, f(\mathbf{y}) = 0 \text{ for all } \mathbf{y} \in \col(A) \right\}.
\]
Because every $\mathbf{y} \in \col(A)$ can be written as $\mathbf{y} = A\mathbf{x}$ for some $\mathbf{x} \in \R^n$, we can rephrase this condition as:
\[
f \in (\col(A))^\circ \quad \Longleftrightarrow \quad f(A\mathbf{x}) = 0 \text{ for all } \mathbf{x} \in \R^n.
\]

Now consider a vector $\mathbf{v} \in \R^m$, and let $f_{\mathbf{v}}$ be the corresponding functional: $f_{\mathbf{v}}(\mathbf{y}) = \mathbf{v}^\top \mathbf{y}$.  
We ask: \textit{When does $f_{\mathbf{v}}$ belong to $(\col(A))^\circ$?}

By the characterization above, this happens precisely when
\[
f_{\mathbf{v}}(A\mathbf{x}) = 0 \quad \text{for all } \mathbf{x} \in \R^n.
\]
Substituting the definition of $f_{\mathbf{v}}$, this becomes:
\begin{equation}
\label{eq:annihilator-condition}
\mathbf{v}^\top (A\mathbf{x}) = 0 \quad \text{for all } \mathbf{x} \in \R^n.
\end{equation}

\subsection*{Step 4: Rewrite the condition using properties of the transpose}

Using the fact that $(XY)^\top = Y^\top X^\top$, we may rewrite this as:
\[
\mathbf{v}^\top A \mathbf{x} = (A^\top \mathbf{v})^\top \mathbf{x}.
\]
Thus, condition~\eqref{eq:annihilator-condition} is equivalent to:
\[
(A^\top \mathbf{v})^\top \mathbf{x} = 0 \quad \text{for all } \mathbf{x} \in \R^n.
\]

In order for this to be true for all $x$ the linear functional $f_{A^T \mathbf{v}}$ must be zero, which iimplies that $A^T v = 0$ which implies that $v \in \nul(A^T)$.


We have shown the following chain of equivalences:
\begin{align*}
\mathbf{v} \in \nul(A^\top) 
&\iff A^\top \mathbf{v} = \mathbf{0} \\
&\iff (A^\top \mathbf{v})^\top \mathbf{x} = 0 \text{ for all } \mathbf{x} \in \R^n \\
&\iff \mathbf{v}^\top A \mathbf{x} = 0 \text{ for all } \mathbf{x} \in \R^n \\
&\iff f_{\mathbf{v}}(A\mathbf{x}) = 0 \text{ for all } \mathbf{x} \in \R^n \\
&\iff f_{\mathbf{v}} \in (\col(A))^\circ.
\end{align*}

Therefore, under the isomorphism $\mathbf{v} \leftrightarrow f_{\mathbf{v}}$, the subspace $\nul(A^\top) \subseteq \R^m$ corresponds, under isomorphism, to the subspace $(\col(A))^\circ \subseteq (\R^m)^*$.

\end{proof}
\end{theorem}

\subsection*{Geometric Interpretation}

This result has a clean geometric meaning:  
a vector $\mathbf{v} \in \R^m$ is orthogonal (with respect to the dot product) to every vector in the column space of $A$ if and only if $A^\top \mathbf{v} = \mathbf{0}$.  
But ``orthogonal to the column space'' is precisely what it means for the functional $f_{\mathbf{v}}(\mathbf{y}) = \mathbf{v} \cdot \mathbf{y}$ to vanish on $\col(A)$ — i.e., to be in the annihilator.  
Thus, in $\R^m$, the annihilator $(\col(A))^\circ$ is naturally identified with the orthogonal complement $\col(A)^\perp$, and we recover the familiar fundamental theorem of linear algebra:
\[
\nul(A^\top) = \col(A)^\perp.
\]







\begin{theorem}
$\nul(A) \cong (\row(A))^\circ$
\begin{proof}
This follows immediately by symmetry.
\end{proof}
\end{theorem}

\noindent\textbf{Conclusion.}  
Via the isomorphism $\psi$ between $\mathbb{R}^k$ \and $(\mathbb{R}^k)^*$, we have the following identities:
\[
\boxed{\nul(A^\top) \cong (\col(A))^\circ}, \qquad
\boxed{\nul(A) \cong (\row(A))^\circ}.
\]



The Annihilator reveals that the four subspaces reflect a fundamental duality.  
The matrix $ A: \mathbb{R}^n \to \mathbb{R}^m $ induces a \textbf{dual map} $ A^*: (\mathbb{R}^m)^* \to (\mathbb{R}^n)^* $, and we have:
\[
\ker(A^*) = (\operatorname{Im} A)^\circ, \qquad \operatorname{Im}(A^*) = (\ker A)^\circ.
\]

When we identify $ (\mathbb{R}^k)^* \cong \mathbb{R}^k $ via the isomorphism $\psi$, these become:
\[
\operatorname{Nul}(A^\top) = \operatorname{Col}(A)^\perp, \qquad \operatorname{Row}(A) = \operatorname{Nul}(A)^\perp.
\]

Hopefully this chapter has been illuminating.



