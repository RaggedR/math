\documentclass{article}
\usepackage{amsmath}
\usepackage{amssymb}

\begin{document}

\title{Converse Proof: Symmetric Idempotent Matrices are Orthogonal Projections}
\author{}
\date{}
\maketitle

\Large

\section*{Theorem}
If a matrix $P \in \mathbb{R}^{n \times n}$ satisfies:
\begin{enumerate}
    \item $P^T = P$ (symmetry)
    \item $P^2 = P$ (idempotence)
\end{enumerate}
then $P$ is an orthogonal projection matrix.

\section*{Proof}

Let $P$ be a matrix satisfying $P^T = P$ and $P^2 = P$. We need to show that $P$ represents an orthogonal projection, which means for any vector $\mathbf{x} \in \mathbb{R}^n$:

\begin{enumerate}
    \item $P\mathbf{x}$ lies in some subspace $W \subseteq \mathbb{R}^n$
    \item $\mathbf{x} - P\mathbf{x}$ is orthogonal to every vector in $W$
\end{enumerate}

\subsection*{Step 1: Define the subspace $W$}

Let $W = \text{range}(P) = \{P\mathbf{x} \mid \mathbf{x} \in \mathbb{R}^n\}$, the column space of $P$.

For any $\mathbf{x} \in \mathbb{R}^n$, $P\mathbf{x} \in W$ by definition of $W$.

\subsection*{Step 2: Show $\mathbf{x} - P\mathbf{x}$ is orthogonal to $W$}

We need to prove that for any $\mathbf{x} \in \mathbb{R}^n$ and any $\mathbf{y} \in W$, the following holds:
\[
(\mathbf{x} - P\mathbf{x}) \cdot \mathbf{y} = 0
\]

Since $\mathbf{y} \in W$, there exists some $\mathbf{z} \in \mathbb{R}^n$ such that $\mathbf{y} = P\mathbf{z}$.

Computing the dot product:
\begin{align*}
(\mathbf{x} - P\mathbf{x}) \cdot \mathbf{y} 
&= (\mathbf{x} - P\mathbf{x})^T (P\mathbf{z}) \\
&= \mathbf{x}^T P\mathbf{z} - (P\mathbf{x})^T P\mathbf{z} \\
&= \mathbf{x}^T P\mathbf{z} - \mathbf{x}^T P^T P\mathbf{z} \\
&= \mathbf{x}^T P\mathbf{z} - \mathbf{x}^T P P\mathbf{z} \quad \text{(since $P^T = P$)} \\
&= \mathbf{x}^T P\mathbf{z} - \mathbf{x}^T P^2\mathbf{z} \\
&= \mathbf{x}^T P\mathbf{z} - \mathbf{x}^T P\mathbf{z} \quad \text{(since $P^2 = P$)} \\
&= 0
\end{align*}

Therefore, $\mathbf{x} - P\mathbf{x}$ is orthogonal to every vector in $W$.

\subsection*{Step 3: Verify the decomposition}

For any $\mathbf{x} \in \mathbb{R}^n$, we can write:
\[
\mathbf{x} = P\mathbf{x} + (\mathbf{x} - P\mathbf{x})
\]
where:
\begin{itemize}
\item $P\mathbf{x} \in W$
    \item $\mathbf{x} - P\mathbf{x}$ is orthogonal to $W$ (as shown in Step 2)
\end{itemize}

This is precisely the defining property of an orthogonal projection.

\subsection*{Step 4: Confirm $P$ is a projection}

We also need to verify that $P$ acts as a projection onto $W$. For any $\mathbf{w} \in W$, there exists $\mathbf{z}$ such that $\mathbf{w} = P\mathbf{z}$. Then:
\[
P\mathbf{w} = P(P\mathbf{z}) = P^2\mathbf{z} = P\mathbf{z} = \mathbf{w}
\]
So $P$ fixes all vectors in $W$, as expected for a projection.

\section*{Conclusion}

We have shown that if $P^T = P$ and $P^2 = P$, then for any vector $\mathbf{x}$:

\begin{itemize}
\item $P\mathbf{x}$ lies in the subspace $W = \text{range}(P)$
    \item $\mathbf{x} - P\mathbf{x}$ is orthogonal to $W$
\end{itemize}

Therefore, $P$ represents an orthogonal projection onto its range.

\boxed{P^T = P \text{ and } P^2 = P \implies P \text{ is an orthogonal projection}}

\textbf{Q.E.D.}

\end{document}

